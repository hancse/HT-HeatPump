\documentclass[a4paper,10pt]{article}
\usepackage[utf8]{inputenc}
\usepackage[english]{babel}
\usepackage[onehalfspacing]{setspace}
\usepackage{float}
\usepackage{amsmath}

\usepackage[nottoc]{tocbibind} %Adds "References" to the table of contents
\usepackage[titletoc]{appendix}

\usepackage{graphicx}
\usepackage{hyperref}
\usepackage{multirow}

%\graphicspath{/home/trung/Pictures/} \usepackage{float}
\hypersetup{
    colorlinks=true,
    linkcolor=blue,
    filecolor=magenta,      
    urlcolor=cyan,
}
%Document title, author and date (empty)
\title{HT Heat Pump WP5 report}
\author{Trung Nguyen}
\date{}

%Beginning of the document
\begin{document}

\maketitle

\tableofcontents

\section{Introduction}
\subsection{Goal}

The goals of WP5 are verifying whether the compressor can be used to build a functioning HT heat pump. These following steps are taken:   

\begin{itemize}
    \item Simulating the work of Heat Pump under reference cases.
    \item reserve...
    \item
    \item
    \item
    \item
\end{itemize}


%-------------------------------------------

\section{Generalized setup of calculation model}

\subsection{Simulation methods}

Building energy simulation is a vast field of research that started on the late 50’s and that is still highly active nowadays. Building energy simulations are mainly used to help taking design decisions, to analyze current designs and to forecast future building energy use. Building energy modelling methods can mainly be divided into three categories: 

\begin{itemize}
    \item White box model (physics-base) 
    \item Black box model (data-driven) 
    \item Grey box (hybrid) 

\end{itemize}
White box model is based on the equations related to the fundamental laws of energy and mass balance and heat transfer. White box models can be differentiated in two types, distributed parameter models and lumped parameter models. Lumped parameter models simplify the description of distributed physical systems into discrete entities that approximate the behavior of a distributed system. The advantage of using lumped models is the decrease on simulation time (Ramallo-González et al.). White box model is of special interest for the design phase as they are used to predict and analyses the performance of the building envelope and building systems.

Black box models are based on the statistical relation between input and output system values. The statistical relation between input and output is based on actual data. The relation between the parameters can differ based on the amount of data and the method used to analyze the relation. Currently, there is a large and active field of research about statistical models that are used on black box models (Coacley et al.). Black box models are of special interest when there is a large amount of actual input and output data available.   

Grey box model is a hybrid model form that aim to combine the advantages of both systems. In order to use them it is necessary to implement some equations and it is also required to have actual data of inputs and outputs. 

\subsection{White box lumped model: RC network} 

The apartment and office building load profile model for this project is to serve as test environment for a heat pump model, what means that the models is intended as a tool to help taking building systems design decisions. The heating needs calculation model implemented for this project is a white box lumped model. Specifically, it is a RC network model consisting of resistances (R) and capacities (C). The RC network model is based on electrical systems analogy. The simulation of thermodynamic systems characterizing building elements as resistances or capacities allows to simplify the model while maintaining a high simulation results accuracy (Bagheri et al., Bacher et al.).   

There are several types of RC models, the most common being 3R4C models and 3R2C models which are applied on the outer and internal wall. For the simulation of simple buildings 3R2C models perform as accurate as more complex 3R4C models (Fraisse et al. ). Considering that one of the objectives for this project is to obtain a fast but accurate simulation of a simple 3R2C network model appeared as starting point. In the 3R2C model two indoor temperature nodes in the dwelling with capacities (usually an air and a wall temperature) and a well-known outdoor temperature are present. Between these 3 temperature nodes 3 heat transfer resistances are present. However, the direct heat transfer between the inner walls and the outdoor air is low. Moreover, uncertainties are present about heat transfer coefficients between walls and indoor air, different indoor temperatures in the house rooms and the ground temperature which deviates from the outdoor temperature. In addition, occupancy behaviour varies strongly. For that reason, we have made a further simplification to a 2R2C model. 
% In section 4 it is shown that this dwelling model delivers a reliable annual energy consumption. 

Assume that the apartment/office building is a single zone, furniture and the surface part of the walls have the same temperature as the air and the wall mass is divided between the air and wall mass. Thus, the capacity of the air node consists of the air capacity, furniture capacity and capacity of a part of the walls. In the resistance $R_{air_{\_}outdoor}$ the influence of heat transmission through the outdoor walls and natural ventilation is considered. 

\subsubsection{Office building}

The heating system layout for office building is shown in figure []:

\begin{figure}[ht]
\centering
\includegraphics[width=1\columnwidth]{pictures/Office_diagram.png}
\caption[Short title]{Heating system layout for office building}
\label{fig:ff2}\end{figure}

The room temperature can be calculated as:

\begin{equation}
C_{air}\frac{dT_{air}}{dt}=\frac{T_{outdoor}-T_{air}}{R_{air_{\_}outdoor}} + \frac{T_{wall}-T_{air}}{R_{air_{\_}wall}} + Q_{inst} + Q_{internal} + CF.Q_{solar}
\end{equation}

\begin{equation}
C_{wall}\frac{dT_{wall}}{dt}=\frac{T_{air}-T_{wall}}{R_{air_{\_}wall}} + (1-CF).Q_{solar}
\end{equation}

 \begin{itemize}
      \item CF: convectional factor.
      \item $Q_{inst}$: delivered heat from heating system (radiator) [W].
      \item $Q_{solar}$: heat from solar irradiation [W].
      \item $T_{air}$: indoor air temperature $^o$C.
      \item $T_{outdoor}$: outdoor temperature $^o$C.
      \item $T_{wall}$: wall temperature $^o$C.
      \item $R_{air_{\_}wall}$: walls surface resistance.
      \item $R_{air_{\_}outdoor}$: outdoor surface resistance.
      \item $C_{air}$: air capacity.
      \item $C_{wall}$: wall capacity.
    \end{itemize}
    

Total heat transfer of solar irradiation through the glass windows. 
\begin{equation}
Q_{solar}=g.\sum(A_{glass}.q_{solar})
\end{equation}

\begin{itemize}
    \item $q_{solar}$: solar radiation on the outdoor walls [$W/m^2$]. 
    \item g: g value of the glass (=ZTA in dutch) [0..1]
    \item A: glass surface [$m^2$].
\end{itemize}

\subsubsection{Apartment Building}

The system layout for an Apartment building is shown in figure [....]. The configuration for space heating is similar with the office. The D.H.W is configured in parallel with space heating.

\begin{figure}[H]
\centering
\includegraphics[width=1\columnwidth]{pictures/Apartment_diagram.png}
\caption[Short title]{Heating system layout for Apartment building.}
\label{fig:ff6}\end{figure}


The energy balance equation for space heating of an apartment is similar with office building (equation [1],[2]) where a single zone will be considered.

\[C_{air}\frac{dT_{air}}{dt}=\frac{T_{outdoor}-T_{air}}{R_{air_{\_}outdoor}} + \frac{T_{wall}-T_{air}}{R_{air_{\_}wall}} + Q_{inst} + Q_{internal} + CF.Q_{solar}\]

\[C_{wall}\frac{dT_{wall}}{dt}=\frac{T_{air}-T_{wall}}{R_{air_{\_}wall}} + (1-CF).Q_{solar}\]

The D.H.W consumption is base on D.H.W user profile and will be discussed in the next chapter.\\
The hot tap water temperature is calculated follow:

% \begin{equation}
%     c\rho V_{HE}\frac{dT_{tap}}{dt}= \alpha_{HE}2\pi r_{HE}\Delta x[T_{HE}(x+\Delta x,t)- T_{tap}] - c\rho F_{tap}(T_{tap} - T_{cold})
% \end{equation}

\begin{equation}
    c\rho V_{HE}\frac{dT_{tap}}{dt}= \alpha_{HE}2\pi r_{HE}\Delta x[T_{HE}(x+\Delta x,t)- T_{tap}] - Q_{D.H.W_{\_}profile}
\end{equation}

% \begin{align*}
% c\rho V_{HE}\frac{dT_{tap}}{dt}= \alpha_{HE}2\pi r_{HE}\Delta x[T_{HE}(x+\Delta x,t)- T_{tap}] - c\rho Ftap(T_{tap} - T_{water})\\

% c\rho V_{HE}\frac{dT_{tap}}{dt}= \alpha_{HE}2\pi r_{HE}\Delta x[T_{HE}(x+\Delta x,t)- T_{tap}] - P_{D.H.W_{\_}profile}
% \end{align*}

\begin{align}
    c\rho\pi r_{HE}^2\Delta x\frac{dT_{HE}(x+\Delta x)}{dt}=c\rho\Dot{m}[T_{HE}(x,t)-T_{HE}(x+\Delta x,t)]-\notag\\
    \alpha_{HE}2\pi r_{HE}\Delta x[T_{HE}(x+\Delta x,t)- T_{tap}]
\end{align}

% \begin{align*}
% c\rho\pi r_{HE}^2\Delta x\frac{dT_{HE}(x+\Delta x)}{dt}=c\rho\Dot{m}[T_{HE}(x,t)-T_{HE}(x+\Delta x,t)]-\\
% \newline \alpha_{HE}2\pi r_{HE}\Delta x[T_{HE}(x+\Delta x,t)- T_{tap}]
% \end{align*}
Where $T_{HE_{\_}in}$ = $T_{hot}$, $T_{HE_{\_}out}$ is the return water temperature from heat exchanger.

$T_{tap}$ is the hot tap water temperature $^oC$, $T_{HE}$ is the heat ex-changer temperature $^oC$, and $Q_{D.H.W_{\_}profile}$  is the hot tap water profile.



\subsection{Radiators equation}
Heat transfer from the radiators and the radiators outlet temperature is described as:
\begin{equation}
\frac{Q_{rad}}{dt}= Q_{in} -Q_{inst} - Q_{loss}
\end{equation}

With $Q_{inst}$ is the heat emission from a radiator. Its depends primarily on the temperature difference between the hot surface and the surrounding air(figure). The heat emission can be calculated[11]: 

\begin{figure}[ht]
\centering
\includegraphics[width=1\columnwidth]{heat_emission_radiator.png}
\caption[Short title]{Heat emission from radiator[11].}
\label{fig:ff2}\end{figure}

\begin{equation}
C_{rad}\frac{dT_{rad}}{dt} = \Dot{m}c\Delta T - Q_{50}\Big(\frac{LMTD}{49.32}\Big)^n
\end{equation}


\begin{align}
C_{rad}\frac{dT_{return}}{dt} = \Dot{m}c(T_{hot}-T_{return})\notag\\
- Q_{50}\Big(\frac{T_{hot}-T_{return}}{ln(T_{hot}-T_{air}) - ln(T_{return}-T_{air})}\frac{1}{49.32}\Big)^n
\end{align}


% \begin{align}
% c\rho r^2\Delta x\frac{dT_{rad1}}{dt} = \Dot{m}c(T_{hot}-T_{rad1}) \notag\\
% - U\frac{l}{\Delta x}A\frac{T_{hot}-T_{rad1}}{ln(T_{hot} -T_{air}) - ln(T_{rad1}-T_{air})}
% \end{align}

% \begin{align}
% c\rho r^2\Delta x\frac{dT_{return}}{dt} = \Dot{m}c(T_{rad1}-T_{return}) \notag\\
% - U\frac{l}{\Delta x}A\frac{T_{rad1}-T_{return}}{ln(T_{rad1} -T_{air}) - ln(T_{return}-T_{air})}
% \end{align}



\begin{itemize}
    \item LMTD: log mean Temperature Difference for the radiators.
    \item Crad: radiator constant.
    \item n: radiator exponent.
    \item $Q_{50}$: heat emission from radiator with temperature difference 50 $^oC$ [W]
    \item n: constant describing the type of radiator (1.33 for standard panel radiators, 1.3 - 1.6 for convectors). 
    \item $T_{rad}$: radiator temperature [$^oC$].
    \item $T_{hot}$: water temperature from buffer tank [$^oC$]. 
    \item $T_{return}$: water return temperature from radiator to the buffer tank [$^oC$].
    \item $T_{air}$: room temperature [$^oC$]. 
\end{itemize}
 Radiators are in general designed for middle panel temperature $70^oC$ - and surrounding air temperature $20^oC$ (difference $50^oC$)

\subsection{Water buffer}

In figure [ ] to minimize the effect of water variation which can lead to the significant reduce in heat pump performance. The middle tank with a smaller size is placed in between Heat Pump and the main buffer.  

\begin{figure}[H]
\centering
\includegraphics[width=1\columnwidth]{pictures/middle_tank.png}
\caption[Short title]{An illustration of heat flow.}
\label{fig:ff6}\end{figure}

The energy balance in the buffer tank can be determined:
% \begin{equation}
% \frac{dQ_{buffer}}{dt}= P_{in} -P_{out} - P_{loss}
% \end{equation}

\begin{equation}
c\rho V_{buffer}\frac{dT_{hot}}{dt}= c\rho \Dot{m}(T_{mid_{\_}tank}-T_{hot}) - c\rho \Dot{m}(T_{hot}-T_{return})
\end{equation}
\\
Where $V_{buffer}$ is the volume of the tank in $m^3$ and $T_{mid_{\_}tank}$ is the temperature of the middle tank $^oC$.

The energy balance in the middle tank:

% \begin{equation}
% \frac{dQ_{mid_{\_}tank}}{dt}= P_{in} -P_{out} - P_{loss}
% \end{equation}

% \begin{equation}
% c.\rho V_{mid_{\_}tank}\frac{dT_{mid_{\_}tank}}{dt}= c\rho \Dot{m}(T_{HP} - T_{mid_{\_}tank}) - c\rho \Dot{m}(T_{mid_{\_}tank} - T_{hot})
% \end{equation}

% or 

\begin{equation}
c.\rho V_{mid_{\_}tank}\frac{dT_{mid_{\_}tank}}{dt}= Q_{HP} - c\rho \Dot{m}(T_{mid_{\_}tank} - T_{hot})
\end{equation}

Where:
\begin{itemize}
    \item $V_{mid_{\_}tank}$ is the volume of the middle tank [$m^3$]
    \item $T_{mid_{\_}tank}$ water temperature inside the middle tank [$^oC$]. 
    \item $Q_{HP}$ Heat output from Heat Pump [W] 
    % \item $T_{HP}$ water temperature output from Heat Pump $^oC$.
\end{itemize}
 

\subsection{Heat Pump model from WP2}

%________________________________________________________
\section{Office building load profile}

For nearly energy-neutral buildings (bijna energieneutrale kantoorgebouwen 'BENG') specific requirements for energy consumption will apply from 2021.The maximum energy requirement for heating, cooling and lighting for utility buildings, is 50 kWh/$m^2$ per year. The number is increased to 65 kWh/$m^2$ per year for healthcare buildings.This excludes the energy consumption for office equipment \href{https://www.energievastgoed.nl/benchmarktool/}{[1]}.\\
The average gas consumption for office building is around 17 $m^3$/$m^2$ (figure 1) for the building with the construction year between 1977 and 1989.The report from ECN has calculated the buildings with an office function cover a total of 87 million m2. This concerns 68,000 buildings in the Netherlands. Gas consumption amounts to 955 million m3 (30 PJ) and electricity to 7800 million kWh (28 PJ). These values are the most accurate available at the moment.  
\href{https://www.energievastgoed.nl/2017/02/14/benchmark-energieverbruik-gebouwen/}{[2],[3]}.

\begin{figure}[ht]
\centering
\includegraphics[width=1\columnwidth]{pictures/ECN.png}
\caption[Short title]{Energy Consumption}
\label{fig:ff1}\end{figure}

% \begin{figure}[ht]
% \centering
% \includegraphics[width=1\columnwidth]{pictures/gas use.png}
% \caption[Short title]{Office building gas consumption per m2 }
% \label{fig:ff1}\end{figure}

The reference building has been selected:

    \begin{itemize}
      \item Construction year: 1977-1989.
      \item Surface $m^2$: 500
      \item Gas consumption: 17 $m^3$/$m^2$
      \item T{\_outside} : out door temperature from NEN5060\texttt{\_}2018 with 1 hour sampling rate for 1 year long.
    \end{itemize}
    
Add plot of load profile .... and validate



% Assume that there is no heating needed, when out side temperature is 15 degree. The heating line is set on 14 $^o$C.
% The office is 500 $m^2$ the gas consumption is around 8500  $m^3$ gas or 82450 kwh (1 $m^3$ equals approximately 9.7 kWh).\\
% The (“graaduren”) $^o$C per hour necessary to achieve the right inside temperature.


% %\[Q_{per{\_}degree} = \frac{97000}{\sum{(14 - T_{outside})}}\] 

% \begin{equation}
% Q_{per{\_}degree}=\frac{97000}{\sum(14 - T_{outside})}
% \end{equation}

% Energy demand profile is calculated with period of 1 hour for the whole year.

% \begin{equation}
% Q_{profile}=Q_{per{\_}degree}*{(14 - T_{outside})}
% \end{equation}

% The energy demand profile is showed in figure 1.

% \begin{figure}[H]
% \centering
% \includegraphics[width=1\columnwidth]{pictures/Q_profile.png}
% \caption[Short title]{Energy Consumption of office building.}
% \label{fig:ff2}\end{figure}

%__________________________________________________

\section{Apartment Building load profile}
\subsection{Heating profile}

The apartment building (Flatwoningen (overig) Gebouwd) build in the period 1965-1974 which represent 1.8 \texttt{\%} of the dutch housing stock [5] (highest number in the flatwoningen category) will be selected. Houses in this category often have 2 to 4 rooms.The average gas consumption for this type of apartment (energy label C) is 829 $m^3$/year with the average usage area is 77 m2 and 2.8 people in the house. Label C has been selected because the focus of heat pump heating will be on the renovated house with good insulation. The property features are showing in figure [  ... ].

\begin{figure}[H]
\centering
\includegraphics[width=1\columnwidth]{pictures/property features.png}
\caption[Short title]{property features [5].}
\label{fig:ff3}\end{figure}


From statista.com (figure 4)one person use approximately 120 liters of water per day therein one third of the water is used for showering and bathing. The rest for washing, toilet and cooking.

\begin{figure}[H]
\centering
\includegraphics[width=1\columnwidth]{pictures/daily water usage.png}
\caption[Short title]{daily water usage by capital.}
\label{fig:ff4}\end{figure}


% If one person use around 60 liters of hot tap water at 40 degree per day. The energy use by one person can be calculated:

% \begin{equation}
% Q = m.c.\Delta T
% \end{equation}
% \\
% 60 liters $\approx$ 60 kg of water.\\
% c = 4.19 kj/kg.K specific heat capacity of water.\\
% $\Delta$T: temperature different: 40 - 10 = 30 $^o$C.\\ 
% \\
% At the results one person will consume 2.095kwh per day for hot tap water. 2.8 people will consume 5.866 kwh per day.\\
According to "VERORDENING (EU) Nr. 814/2013 VAN DE COMMISSIE van 2 augustus 2013". The hot water capacity M profile (will be discussed in next chapter) has a sum of energy use for tap water per day 5,845 kwh.

\begin{figure}[H]
\centering
\includegraphics[width=1\columnwidth]{pictures/hot tap water.png}
\caption[Short title]{example hot tap water profile.}
\label{fig:ff5}\end{figure}


% \begin{figure}[H]
% \centering
% \includegraphics[width=1\columnwidth]{pictures/Tap_water_perm2.png}
% \caption[Short title]{Energy Consumption of the Apartment}
% \label{fig:ff6}\end{figure}.



% An average energy use for heating is 5907.875 kwh per apartment. The heat demand profile for apartment building will be calculated follow equation (2) and (3). 

% \begin{figure}[H]
% \centering
% \includegraphics[width=1\columnwidth]{pictures/Apartment_profile.png}
% \caption[Short title]{Energy Consumption of the Apartment.}
% \label{fig:ff6}\end{figure}




%________________________________________________________
\subsection{Hot tap water profile}
 
 
A normal dutch house using CW label 3 or 4 (Appendix A). Therefore capacity profile M will be selected as a reference.\\
Capacity profile definition:
Annex III of the EU regulation[3] describes a capacity profile for a 24-hour measurement cycle intended for checking compliance with the requirements.\\
This capacity profile is as follows:

\begin{itemize}
      \item No hot water consumption from 9:35 PM to 7 AM.
      \item From 07:00 to 21:30 hot water consumption according to a precisely defined tap profile.
    \end{itemize}
An example calculation is shown below: 

\begin{figure}[H]
\centering
\includegraphics[width=0.6\columnwidth]{pictures/Tap_water example.png}
\caption[Short title]{Tap water example [4].}
\label{fig:ff7}\end{figure}
At 7.00 $Q_{tap}$ is 0.105 kwh with the flow rate at 3 liters  per minutes and temperature added is 25 degree.

\[m = \frac{3600.Q}{c.\Delta T}\]

m = 6 liters(kg), c=4.19 kj/kg.K. Therefore 2 minutes of tap water has been used at 7.00 AM

The hot tap water profile M for 1 day is showing in figure 10:

\begin{figure}[H]
\centering
\includegraphics[width=1\columnwidth]{pictures/hot_tap_water_profile_M.png}
\caption[Short title]{hot tap water profile M.}
\label{fig:ff8}\end{figure}

\subsection{Simulation profile for the entire building}

The simulation is base on the assumption of the apartment building with 24 residential apartments.
The hot tap water demand for 24 apartments will be calculated base on the calculation rules public in kennisbank isso 55.

\begin{figure}[H]
\centering
\includegraphics[width=1\columnwidth]{pictures/Calculation rules for hot tap water demand for apartment buildings.png}
\caption[Short title]{Calculation rules for hot tap water demand for apartment buildings[6].}
\label{fig:ff9}\end{figure}
The instantaneous hot water requirement for an hour is:

\begin{figure}[H]
\centering
\includegraphics[width=0.5\columnwidth]{pictures/tap water demand in an hour.png}
\caption[Short title]{Calculation rules for hot tap water demand for an hour[6].}
\label{fig:ff10}\end{figure}

\begin{equation}
V_{WW;60} = 243 + 67.9\sqrt{n} + 12.8n
\end{equation}

$V_{WW;60}$: volume flow of hot water in 60 minutes
n: number of apartments

The hot tap water profile for building apartment are shown below:

%---------------------------------------------------------------------
%Example of the multicolumn command
\begin{table}[h!]
\centering
\begin{tabular}{|p{3cm}|p{3cm}|}
\hline
\multicolumn{2}{|c|}{1 day hot tap water usage} \\
\hline
Time& $Q_{tap}$ (kwh)\\ % & Use periods(hour)
 \hline
 07:00	& 38.64   \\ % &1.1
 08:00  & 10.08   \\ % &0.8
 09:00	& 5.04    \\ % &0.4
 10:00  & 2.52    \\ % &0.1
 11:00  & 5.04    \\ % &0.4
 12:00  & 7.56    \\ % &0.2
 13:00  & 0         \\ %&0
 14:00  & 2.52   \\ %& 0.2
 15:00  & 2.52   \\ % & 0.2
 16:00  & 2.52   \\ % & 0.2
 17:00  & 0        \\ % & 0
 18:00  & 7.56   \\ % & 0.4
 19:00  & 2.52  \\ %  & 0.2
 20:00  & 17.64 7\\ % & 0.3
 21:00  & 36.12 \\ %  & 0.9
 22:00  & 0        \\ % & 0

  \hline
 \end{tabular}
 \caption{Hot water profile}

 \end{table}


\begin{figure}[H]
\centering
\includegraphics[width=1\columnwidth]{pictures/tap water profile of 24 apartments.png}
\caption[Short title]{Hot tap water use of entire apartment.}
\label{fig:ff11}\end{figure}

% The heat demand for the building:

% \begin{equation}
% Q_{building} = Q_{hotwater} + Q_{heating}
% \end{equation}

% \begin{figure}[H]
% \centering
% \includegraphics[width=1\columnwidth]{pictures/Q_building profile.png}
% \caption[Short title]{Energy demand for an Apartment.}
% \label{fig:ff12}\end{figure}

\subsection{BENG and load profile correction}

Total energy consumption for D.H.W with profile M [4] (Appendix A) is 5,845 kwh per day. The annual consumption is therefore approximately 2134 kwh. The value is much higher than the indicated value ( approximately 1600 kwh) for 77m2 apartments from NTA8800 [7], figure 13.

\begin{figure}[H]
\centering
\includegraphics[width=1\columnwidth]{pictures/NTA_8800_DHW.png}
\caption[Short title]{Hot tap water use per year (kwh/$m^2$) [7].}
\label{fig:ff13}\end{figure}

According to NEN 12831-3 [10], Method for calculation of the design heat load - Part 3:  Domestic hot water systems heat load and characterisation of needs. The value of volume of hot water per day ($V_{W;P;day}$) can be calculated based on the number of equivalent persons (adults) $n_{P,eq}$.
\vspace{2mm}
\\
\textbf{Apartment dewellings}.\\
The area is used to calculate $n_{P,eq}$,max as follow: 
\begin{equation}
    n_{P,eq,max}= 1.75 - \begin{cases}
			1, & \text{if $A_{h} < 10 m^2$}\\
            0.01875\cdot(50 - A_h), & \text{if $10 m^2 <A_{h} < 50 m^2$}\\
            0.035\cdot A_h, & \text{if $A_{h} > 50 m^2$}
		 \end{cases}
\end{equation}\\
The total number of equivalent persons is defined by formula:


\begin{equation}
    n_{P,eq}= 1.75 + \begin{cases}
			0.3\cdot n_{P,eq,max}, & \text{if $n_{P,eq,max} < 1.75$}\\
            0.3\cdot(n_{P,eq,max}- 1.75), & \text{if $n_{P,eq,max} \geq 1.75$}
		 \end{cases}
\end{equation}\\
For the residential case and at the level of one dwelling, requirements can be expressed by formula:


\begin{equation}
   V_{W,P,day} = min\Big(x;\Big(y\cdot \frac{A_h}{n_{P,eq}}\Big)\Big)
\end{equation}\\
Where:\\ 
$A_{h}$: habitable area.\\
$n_{P},e_{q}$: number of equivalent persons used for calculating the D.H.W requirements.\\
$n_{P},e_{q,max}$: maximum number of equivalent persons corresponding to the part of the group supplied by the same D.H.W transmitter (individual or attached house and collective housing).\\
The default values for x, and y are:
x = 40,71
y = 3,26 [10]

Applied equations (13) and (14) for the 77m2 apartment dwelling.$n_{P},e_{q}$ = 1.4665.

Compare the assumption of define cases with 2.8 people in 77 $m^2$ there are more likely that the EU-M capacity profile was define for more than 1.4 people,


Check--------------------------------------

According to NEN NTA2020 - 13.3.2.1 / 856kwh/jaar/person = 856*2.8= 2396.8

%________________________________________________________
\section{Associated operating conditions compressor}



%________________________________________________________

\section{ Package of requirements and compressor test plan}

%________________________________________________________


\section{Overview of components with substantiation}






\medskip

%Bibliographic references
\begin{thebibliography}{9}

%______________________

\bibitem{1} 
\texttt{https://www.energievastgoed.nl/benchmarktool/}

%______________________

\bibitem{2} 
\texttt{https://www.energievastgoed.nl/2017/02/14/benchmark-energieverbruik-gebouwen/}


%______________________

\bibitem{3} 
\texttt{Ontwikkeling energiekentallen utiliteitsgebouwen,Een analyse van 24 gebouwtypen in de dienstensector en 12 industriële sectoren, J.M. Sipma, M.D.A. Rietkerk, Januari 2016, ECN-E--15-068}

%______________________


\bibitem{4} 
\texttt{VERORDENING (EU) Nr. 814/2013 VAN DE COMMISSIE
van 2 augustus 2013
tot uitvoering van Richtlijn 2009/125/EG van het Europees Parlement en de Raad wat eisen inzake
ecologisch ontwerp voor waterverwarmingstoestellen en warmwatertanks betreft}

%______________________


\bibitem{5} 
\texttt{Voorbeeldwoningen 2011 Bestaande bouw}

%______________________

\bibitem{6} 
\texttt{KENNISBANK.ISSO.NL, ISSO-publicatie 55,  01-06-2013, ISBN: 978-90-5044-250-3}


%______________________
\bibitem{7} 
\texttt{Nederlandse technische afspraak, NTA 8800, Energieprestatie van gebouwen - Bepalingsmethode, Vervangt NTA 8800:2019-06, ICS 91.120.10; 91.140.30, juli 2020}


%______________________

\bibitem{8} 
\texttt{ECOdesign, meer dan een
label, 20 TVVL Magazine | 03 | 2014 REGELGEVING}

%______________________


\bibitem{9} 
\texttt{\href{https://eur-lex.europa.eu/legal-content/NL/TXT/HTML/?uri=CELEX:32013R0812&from=EN}{https://eur-lex.europa.eu}}

%______________________


\bibitem{10} 
\texttt{NEN-EN 12831-3, Energy performance of buildings - Method for
calculation of the design heat load - Part 3:
Domestic hot water systems heat load and
characterisation of needs, Module M8-2, M8-3}

\bibitem{11} 
\texttt{https://www.engineeringtoolbox.com/heat$-$emission$-$radiators$-$d$_{\_}$272.html}


\end{thebibliography}

%%____________ Appendix_______________________


\begin{appendices}
  \section{Hot tap water profile}

Since september 26, 2015, the Ecodesign [8] and energy labeling guidelines,also apply to appliances for the production of domestic hot water. Devices that do not comply with these guidelines may no longer be sold. In figure 5 the tap water profile label has been highlighted with red square.\\
 

 \begin{figure}[H]
\centering
\includegraphics[width=1\columnwidth]{pictures/energy label.png}
\caption[Short title]{Energy label[8]}
\label{fig:ff7}\end{figure}
The hot water classes are defined in the EU regulation with different category label. For example hot water class with label "XL" must be able to supply 18 kWh of hot water per day (1 kWh corresponds to 64.8 MJ heat). This EU regulation also stipulates what conditions hot tap water class must be met.
Figure 8 show that tap water profile for showering can only be selected from label S to XXL.

 \begin{figure}[ht]
\centering
\includegraphics[width=1\columnwidth]{pictures/tap profile.png}
\caption[Short title]{hot water profile labels[8].}
\label{fig:ff8}\end{figure}

  
\begin{figure}[H]
\centering
\includegraphics[width=1\columnwidth]{pictures/Profile_M1.png}
\caption[Short title]{Capaciteitsprofielen van waterverwarmingstoestellen.}
\label{fig:ff15}\end{figure}
  
\begin{figure}[H]
\centering
\includegraphics[width=1\columnwidth]{pictures/Profile_M2.png}
\caption[Short title]{Capaciteitsprofielen van waterverwarmingstoestellen.}
\label{fig:ff16}\end{figure}

\begin{figure}[H]
\centering
\includegraphics[width=1\columnwidth]{pictures/Profile_M3.png}
\caption[Short title]{Capaciteitsprofielen van waterverwarmingstoestellen.}
\label{fig:ff17}\end{figure}

%____________________________________________________

\section{Calculation methods}

Hot tap water profile for apartments building with 24 residential houses.
From equation (4):

\[V_{WW;60} = 243 + 67.9\sqrt{n} + 12.8n\]

Applied for n = 4 and n = 24

\[ratio =\frac{V_{WW;60;24}}{V_{WW;60;1}}\]
\[f_{new} =f.ratio\]
$f_{new}$: new water flow rate.\\
Look at figure 9 for an example.The amount of water use (litter)

\[m = \frac{3600.Q_{tap}}{c.\Delta T}\]

The time period that water has been used (hour)
\[water_{usetime} = \frac{m.24}{f_{new}.60}\]

Energy use per hour for 24 apartments:

\[Q_{tap/perhour} = 24.Q_{tap}.water_{usetime}\]

\[Q_{building} = Q_{hotwater} + Q_{heating}\]

with $Q_{tap/perhour} = Q_{hotwater}$


%_______________________________________________________________________________
\section{Office building Construction}




Volume floor and internal walls construction:
\begin{equation}
V_{internal_{\_}mass}=A_{internal_{\_}mass}.th_{internal_{\_}mass}
\end{equation}
\\
$V_{internal_{\_}mass}$: volume floor and internal walls [$m^3$].\\
$A_{internal_{\_}mass}$: floor and internal walls surface [$m^2$].\\
$th_{internal_{\_}mass}$: construction thickness [m].

Ventilation volume air flow [$m^3/s$]:

\begin{equation}
q_{V}=\frac{n.V_{dwelling}}{3600}
\end{equation}
\\
n: ventilation air change per hour.\\  
$V_{dwelling}$: internal volume $m^3$.\\

Ventilation, mass air flow [kg/s]

\begin{equation}
q_{m}=q{V}.\rho_{air}
\end{equation}
\\
$\rho_{air}$: air density [$kg/m^3$].

Resistance indoor $air_{wall}$:

\begin{equation}
R_{air{\_}wall}=\frac{1}{A_{internal{\_}mass}.\alpha_{internal{\_}mass}}
\end{equation}
\\
Thermal transmittance indoor air-facade [$W/m^2$]: 

\begin{equation}
U=\frac{1}{A_{alpha_i{\_}_facade} + Rc_{facade} + \frac{1}{\alpha_{e_{\_}facade}}}
\end{equation}

Resistance indoor air-outdoor air:

\begin{equation}
R_{air_{\_}outdoor}=\frac{1}{A_{facade}.U + A_{glass}.U{glass}+q{m}.c_{air}}
\end{equation}

Indoor air and walls capacity:

\begin{equation}
C_{air}=\frac{\rho_{internal_{\_}mass}.c_{internal_{\_}mass.V_{internal_{\_}mass}}}{2 + \rho_{air}.c_{air}.V_{dwelling}}
\end{equation}

\begin{equation}
C_{wall}=\frac{\rho_{internal_{\_}mass}.c_{internal_{\_}mass.V_{internal_{\_}mass}}}{2}
\end{equation}




  
\end{appendices}


\end{document}

% The D.H.W does not match between profile and year annual consumption.
